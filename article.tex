\begin{flushleft}
\setlength{\parindent}{0pt}
\begin{center}
  ОТВЕТЫ, УКАЗАНИЯ, РЕШЕНИЯ  
\end{center}

\begin{multicols}{2}
средним геометриеским имеем 

\(A\ge\frac{a+b+\frac{b^2}{4a}}{b-a} = \frac{4a^2+4ab+b^2}{4a(b-a)} = \frac{9a^2+6ax+x^2}{4ax}\) =

\(\frac{3}{2}+\frac{9a^2+x^2}{4ax} \ge \frac{3}{2} + \frac{\sqrt{9a^2\cdot x^2}}{2ax} = \frac{3}{2} + \frac{3}{2} = 3\), 

причем равенство \(A = 3\) достиаетс, если \(c = \frac{b^2}{4a}\) и \(x = 3a\), т.е. при \(b = c = 4a\).

Следовательно, данное выражение принимает свое наименьшее значение, равное трем, когда \(f(x) = ax^2 + 4ax + 4a = a(x + 2)^2\), где \(a\) - произвольное положительное число.

\textit{Замечание}. 
Пусть \(g(t) = \frac{(t+2)^2}{4(t-1)}\). Нетрудно проверить, что \(g(\frac{b}{a}) = \frac{4a^2+4ab+b^2}{4a(b-a)}\). Следовательно, наименьшее значение A можно найти, исследовав функцию \(g(t)\) на экстремум при \(t>1\)

\textbf{6.}
Обозначим \(\Vec{AD} = \Vec{a}, \Vec{AB} = \Vec{b}, \Vec{DE} = \Vec{c}\) и \(\Vec{BF} = \Vec{d}\) (рис.14).

%нарисовать рисунок 14
\begin{tikzpicture}
\draw[->] (3.0,0.0) -- (1.0,2.0) %AB
\draw[->] (3.0,0.0) -- (4.7,2.3) %AD
\draw (3.0,0.0) -- (2.5,3.0) %AC
\draw (1.0,2.0) -- (2.5,3.0) %BC
\draw (4.7,2.3) -- (2.5,3.0) %DC
\draw (3.0,0.0) -- (3.6,2.65) %AE
\draw[->] (4.7,2.3) -- (3.6,2.65) %DE
\draw[->] (1.0,2.0) -- (2.0,2.666666) %BF
\draw (4.7,2.3) -- (2.0,2.666666) %DF
\draw[dash dot] (3.6,2.65) -- (1.0,2.0) %EB
\draw[dash dot] (3.0,0.0) -- (2.0,2.666666) %AF
\draw (3.8, 2.44) -- (3.75,2.25)
\draw (3.5,2.28) -- (3.75,2.25)

\node at (3.2,0.0) {A}
\node at (0.8,2.0) {B}
\node at (4.9,2.3) {D}
\node at (2.6,3.2) {C}
\node at (3.8,2.85) {E}
\node at (1.8, 2.8) {F}
\node at (4.0, 0.9) {\(\Vec{a}\)}
\node at (1.8,0.8) {\(\Vec{b}\)}
\node at (1.4,2.65) {\(\Vec{d}\)}
\node at (4.2,2.7) {\(\Vec{c}\)}
\end{tikzpicture}

\textit{Рис.14}

Тогда \(\Vec{DF} = \Vec{b} + \Vec{d} - \Vec{a}\) , \(\Vec{AE} = \Vec{a} + \Vec{c}\) , \(\Vec{AF} = \Vec{b} + \Vec{d}\) , \(\Vec{BE} = \Vec{a} + \Vec{c} - \Vec{b}\). По условию \(DF \bot AE\) и \(AD \bot DE\) , поэтому 

\((\Vec{b}+\Vec{d}-\Vec{a})\cdot(\Vec{a}+\Vec{c}) = 0\), \((\Vec{b}+\Vec{d})\cdot(\Vec{a}+\Vec{c}) - |\Vec{a}|^2 = 0\).

Так как \(AB \bot BF\), то 

\(\Vec{AF}\cdot\Vec{BE} = (\Vec{b} + \Vec{d})\cdot(\Vec{a} + \Vec{c}-\Vec{b}) = (\Vec{b} + \Vec{d})\cdot(\Vec{a} + \Vec{c}) - |\Vec{b}|^2\).

Отсюда в силу условия \(|\Vec{b}| = \Vec{a}\) следует, что \(\Vec{AF}\cdot\Vec{BE} = 0\), т.е. \(AF \bot BE\).

\textbf{7.} 
Выберем в ожерелье какой-нибудь кубик и отметим его номером 1. Затем занумеруем остальные кубики по порядку, двигаясь вдоль нити в одном из двух возможных направлений. В кубике с номером \(n\) обозначим через \(n_{1}\) ту вершину, которая примыкает к предыдущему кубику, а через \(n_{2}\) - вершину, а через \(n_{2}\) - вершину, примыкающую к следующему кубику(рис.15). Так как ожерелье замкнутое, то первый кубик следует за \(N^3\)-м.

\begin{multicols}{2}

%нарисовать рис.15 и рис.16
\begin{tikzpicture}[scale=0.8]
%1куб
\draw (1.0,1.0) -- (2.0,1.0)
\draw (1.0,1.0) -- (1.0,2.0)
\draw (1.0,2.0) -- (2.0,2.0)
\draw (2.0,2.0) -- (2.0,1.0)
\draw (2.0,1.0) -- (2.25,1.25)
\draw (2.25,1.25) -- (2.25,2.25)
\draw (2.0,2.0) -- (2.25,2.25)
\draw (1.0,2.0) -- (1.25,2.25)
\draw (1.25,2.25) -- (2.25,2.25)
%2куб
\draw (2.25,2.25) -- (2.25,3.25)
\draw(2.25,2.25) -- (3.25,2.25)
\draw (3.25,2.25) -- (3.25,3.25)
\draw (2.25,3.25) -- (3.25,3.25)
\draw (3.25,3.25) -- (3.5,3.5)
\draw(2.25,3.25) -- (2.5,3.5)
\darw (3.25,2.25) -- (3.5,2.5)
\draw(3.5,2.5) -- (3.5,3.5)
\draw (2.5,3.5) -- (3.5,3.5)
%диагональная линия
\draw (0.5,0.5) -- (4.0,4.0)
\draw[dash dot] (0.0,0.0) -- (0.5,0.5)
\draw[dash dot] (4.0,4.0) -- (4.5,4.5)
%стрелки 
\draw[->] (0.0,1.25) -- (1.0,1.0)
\draw[->] (2.0,0.75) -- (1.0,1.0)

\draw[->] (1.25,2.5) -- (2.25,2.25)
\draw[->] (3.25,2.0) -- (2.25,2.25)

\draw[->] (2.5,3.75) -- (3.5,3.5)
\draw[->] (4.5,3.25) -- (3.5,3.5)
%подписи
\node at (1.2,1.8) {\(n\)}
\node at (2.75,3.05) {\(n + 1\)}
\node at (0.3,1.4) {\(n_{1}\)}
\node at (1.8,0.5) {\((n-1)_{2}\)}
\node at (1.5,2.7) {\(n_{2}\)}
\node at (3.1,1.7) {\((n + 1)_{1}\)}
\node at (4.3, 3.0) {\((n + 1)_{2}\)}
\end{tikzpicture}

%рис.16
\begin{tikzpicture}[scale=0.85]
\draw (1.0,1.0) -- (3.0,1.0)
\draw (1.0,1.0) -- (1.0,3.0)
\draw (1.0,3.0) -- (3.0,3.0)
\draw (3.0,1.0) -- (3.0,3.0)
\draw (1.0,3.0) -- (1.75,3.75)
\draw (3.0,1.0) -- (3.75,1.75)
\draw (3.0,3.0) -- (3.75,3.75)
\draw (1.75,3.75) -- (3.75,3.75)
\draw (3.75,1.75) -- (3.75,3.75)
\draw[dash dot] (1.0,1.0) -- (1.75,1.75)
\draw[dash dot] (1.75,1.75) -- (1.75,3.75)
\draw[dash dot] (1.75,1.75) -- (3.75,1.75)
%оси
\draw[->] (3.0,1.0) -- (4.0,1.0)
\draw[->] (1.0,3.0) -- (1.0,4.0)
\draw[dash dot][->] (1.75,1.75) -- (2.5,2.5)
%подписи
\node at (0.8,0.8) {0}
\node at (4.0,0.8) {\(x\)}
\node at (0.8,4.0) {\(z\)}
\node at (2.7,2.3) {\(y\)}
\node at (0.8,1.2) {1}
\node at (0.7,1.8) {\(N^3\)}
\node at (-0.1,2.8) {\(N^3 - N + 2\)}
%маленькие кубы
\draw (1.4,1.0) -- (1.4,3.0)
\draw (1.6,1.2) -- (1.6,3.2)
\draw (1.4,1.0) -- (1.6,1.2)
\draw (1.4,3.0) -- (1.6,3.2)
\draw (1.6,3.2) -- (1.2,3.2)
\draw (1.0,1.4) -- (1.4,1.4)
\draw (1.4,1.4) -- (1.6,1.6)
\draw (1.0,1.8) -- (1.4,1.8)
\draw (1.4,1.8) -- (1.6,2.0)
\draw (1.0,2.6) -- (1.4,2.6)
\draw (1.4,2.6) -- (1.6,2.8)
\end{tikzpicture}
\end{multicols}

\begin{multicols}{2}
\begin{flushleft}
\textit{Рис.15}
\end{flushleft}
\columnbreak
\begin{center}
\textit{Рис.16}
\end{center}
\end{multicols}

a) Докажем, что при четном \(N\) требуемая упаковка возможна. Выберем систему координат, направив оси вдоль ребер коробки и взяв в качестве единицы длины ребро кубика (рис.16).

Составим столбец высотой \(N\) из кубиков с номерами 1, \(N^3\), \(N^3 - 1\), \dots, \(N^3 - N^2 + 2\), поместив вершину \(1_{1}\) в точку с координатами \((1,0,1)\), а вершину \(1_{2}\) в точку с координатами \(0,1,0\). 

Заметим, что последняя вершина этого столбца, т.е. \((N^3 - N^2 + 2)_{1}\), имеет координаты \((0,1,N)\). Оставшиеся \(N^3 - N\) кубиков будем укладывать в виде <<змеек>>.

1) Первый (нижний) слой - рис.17. В клетках проставлены номера кубиков. Укладывать слой начинаем с кубика 2.
\begin{center}
\begin{tikzpicture}[scale=0.9]
%------
\draw (0.0,0.0) -- (5.7,0.0)
\draw (0.0,1.0) -- (5.7,1.0)
\draw (0.0,2.0) -- (5.7,2.0)
\draw (0.0,3.0) -- (5.7,3.0)
\draw (0.0,4.0) -- (5.7,4.0)
\draw (0.0,5.0) -- (5.7,5.0)
\draw (0.0,6.0) -- (5.7,6.0)
%||||
\draw (0.0,0.0) -- (0.0,6.0)
\draw (1.0,0.0) -- (1.0,6.0)
\darw (2.2,0.0) -- (2.2,6.0)
\draw (3.4,0.0) -- (3.4,6.0)
\draw (5.7,0.0) -- (5.7,6.0)

% Горизонтальные стрелки
\draw[->] (0.8,5.5) -- (1.1,5.5)
\darw[->] (2.1,5.5) -- (3.8,5.5)
\draw[->] (3.3,4.5) -- (2.3,4.5)
\draw[->] (2.3,3.5) -- (2.7,3.5)
\draw[->] (3.3,0.5) -- (2.3,0.5)
\draw[->] (0.8,5.5) -- (1.15,5.5)

% Вертикальные стрелки
\draw[->] (0.5,0.7) -- (0.5,1.3)
\draw[->] (0.5,1.7) -- (0.5,5.3)
\draw[->] (4.5,5.2) -- (4.5,4.8)
\draw[->] (1.6,4.3) -- (1.6,3.7)
\draw[->] (4.5,1.3) -- (4.5,0.7)

% Добавление текста в ячейки
\node at (0.5,5.5) {\( N^2 \)}
\node at (1.6,5.5) {\(N + 1\)}
\node at (4.5,5.5) {\(2N - 1\)}
\node at (1.6,4.5) {\(3N - 2\)}
\node at (1.6,3.5) {\(3N - 1\)}
\node at (4.5,4.5) {\(2N\)}
\node at (3.0,3.5) {. .}
\node at (1.6,2.5) {. .}
\node at (1.6,1.5) {. .}
\node at (1.6,0.5) {\(N^2\)}
\node at (0.5,0.5) {1}
\node at (0.5,1.5) {2}
\node at (2.8,1.5) {. . . . .}
\node at (2.8,2.5) {. . . . .}
\node at (4.5,2.5) {. .}
\node at (4.5,1.5) {\(3N^2 - N + 1\)}
\node at (4.5,0.5) {\(N^2 - N + 2\)}
\end{tikzpicture}
\end{center}
\textit{Рис.17}

Вершина \((N^2)_{2}\) имеет координаты \((1,0,1)\), т.е. \((N^2)_{2} = 1_{1}\). 

2) Второй слой - рис.18. Здесь вершины \((2N^2 - 1)_{2}\) и \((N^3)_{1}\) имеют координаты \((0,1,2)\), т.е. \((2N^2 - 1)_{2} = (N^3)_{1}\) 

% нарисовать рис. 18
\begin{center}
\begin{tikzpicture}[scale=0.9]
%-----
\draw (0.0,0.0) -- (8.0,0.0)
\draw (0.0,1.0) -- (8.0,1.0)
\draw (0.0,2.0) -- (8.0,2.0)
\draw (0.0,4.0) -- (8.0,4.0)
\draw (0.0,5.0) -- (8.0,5.0)

%|||||
\draw (0.0,0.0) -- (0.0,5.0)
\draw (1.4,0.0) -- (1.4,5.0)
\draw (2.6,0.0) -- (2.6,5.0)
\draw (3.5,0.0) -- (3.5,5.0)
\draw (5.7,0.0) -- (5.7,5.0)
\draw (8.0,0.0) -- (8.0,5.0)

%горизонтальные стрелки
\draw[->] (2.8,0.5) -- (5.5,0.5)
\draw[->] (4.0,1.2) -- (3.2,1.2)

%вертикальные стрелки
\draw[->] (0.7,2.1) -- (0.7,3.9)
\draw[->] (4.65,3.8) -- (4.65,2.2)
\draw[->] (6.8,2.2) -- (6.8, 3.8)
\draw[->] (6.8,0.8) -- (6.8,1.2)

\node at (0.7,0.5) {\(N^2\)}
\node at (0.7,1.5) {\(2N^2 - 1\)}
\node at (0.7,4.3) {.}
\node at (0.7,4.6) {.}
\node at (2.0,0.5) {\(N^2 + 1\)}
\node at (2.0,1.3) {.}
\node at (2.0,1.7) {.}
\node at (2.0,2.4) {.}
\node at (2.0,2.7) {.}
\node at (2.0,3.0) {.}
\node at (2.0,3.3) {.}
\node at (2.0,3.6) {.}
\node at (2.0,4.3) {.}
\node at (2.0,4.6) {.}
\node at (4.6,1.5) {\(N^2 + 3N - 3\)}
\node at (4.6,4.5) {\(N^2 + 2N - 1\)}
\node at (3.0,1.3) {.}
\node at (3.0,1.7) {.}
\node at (3.0,2.4) {.}
\node at (3.0,2.7) {.}
\node at (3.0,3.0) {.}
\node at (3.0,3.3) {.}
\node at (3.0,3.6) {.}
\node at (3.0,4.3) {.}
\node at (3.0,4.6) {.}
\node at (6.8,0.5) {\(N^2 + N - 1\)}
\node at (6.8,1.5) {\(N^2 + N\)}
\node at (6.8,4.5) {\(N^2 + 2N - 2\)}
\end{tikzpicture}
\end{center}
\textit{Рис.18}

3) В третьем слое расположение кубиков с номерами \(2N^2, \dots, 3N^2 - 2\) повторяет расположение кубиков с номерами \(2, \dots, N^2\) в первом слое, и т.д.

Заметим, что в каждом слое координаты вершины <<1>> кубика из столбца совпадают с координатами вершины <<2>> последнего кубика из змейки. Следовательно, в \(N\)-м слое координаты вершин \((N^3 - N + 2)_{1}\) и \((N^3 - N + 1)_{2}\) совпадают. Что и требовалось доказать.

б) Если ожерелье упаковано в коробку, то вершины <<1>> и <<2>> любого кубика имеют различные по четности абсциссы. Значит, сумма этих двух координат для каждого кубика - нечетное число. Следовательно, в случае \(N = 2k + 1\) сумма всех абсцисс отмеченных вершин - также нечетное число. Но каждая абсцисса повторяется дважды: для \(n_{2}\) и для \((n + 1)_{1}\). Значит, указанная сумма должна быть четной. Таким образом, при нечетном \(N\) упаковать ожерелье в коробку невозможно.
\end{multicols}
\end{flushleft}
\end{document}